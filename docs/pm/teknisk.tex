\documentclass{TDP003mall}

\usepackage{listings}
\usepackage{float}
\usepackage{array}
\usepackage{color,colortbl}
\usepackage{hyperref}
\usepackage{dirtytalk}
\usepackage{graphicx}

\definecolor{light-gray}{gray}{0.95}
\definecolor{tableshade}{gray}{0.9}

\lstset{
xleftmargin = 0.5cm,
xrightmargin = 0.5cm,
framexleftmargin = 0.5em,
framextopmargin = 0.5em,
framexbottommargin = 0.5em,
frame=single
}

\graphicspath{ {../../images/} }

\newcommand{\version}{Version 1.0}
\author{Mats Eriksson, \url{mater307@student.liu.se}}
\title{Technical Promemoria \\ TDP028}
\date{2019-01-04}
\rhead{Mats Eriksson}

\begin{document}
\projectpage
%##############################################################################
\section{Revisionhistory}
%##############################################################################
\begin{table}[!h]
\begin{tabularx}{\linewidth}{|l|X|l|}
\hline
Ver. & Revision summery & Date \\\hline
1.0 & First draft & 190105 \\\hline
\end{tabularx}
\end{table}

%##############################################################################
\section{Introduction}
%##############################################################################
The main idea for this application is a mix between Treasure hunt and Quiz walk. The location
system in the mobile platform will trigger the next clue/quiz when the user arrives there.
The main focus will be at playing the game at different sightseeing places.
While the game type Treasure hunt or Scavenger hunt has been popular game
used for decades, usually in paper form. There has not been a digital version mainly
due to technical difficulties. When the location system was implemented in the mobile
platforms the popularity of these game types increased rapidly. The concept of these types of games
can change, but the main idea is still the same, try to find the hidden object by answering
a series of clues.

%##############################################################################
\section{Educational purpose}
%##############################################################################
By adding educational value into the game, certain historical or cultural places use
it as a guide. Especially museums offer this type of games as an alternative way to guide their
customer. This actively involves the customer, increase the visitor experience and the learning get
more engaged\cite{daphne15}.
While the study show that this type of interaction can improve the learning, it also lacks
in other fields. Especially, feedback to the teacher is more or less none. Future system
could include a real-time learning analytics model (xAPI), which will help the teacher/supervisor on
the endgame assessment. Still traditional assessment will be required, but a xAPI model will
be an effective tool to improve the game and in this case more importantly they way the students
are learning \cite{victor18}.

%##############################################################################
\section{The market today}
%##############################################################################
There are many applications which purpose is only for a certain route or at a specific place.
But there is also an increasing area of normal treasure hunt games, \textit{Geocaching} being one example.
One central problem of treasure hunt is often needs a real physical object. The goal
object can disappear or been removed to a safer location after some time. The Museums often need
to do repair work which makes the place unaccessible. New games are actively
trying to change this object into virtual objects, which reduce cost and better maintenance of
the game\cite{zsolt12}\cite{ridho17}.
Even games like \textit{Pokeman Go} or other location-based games would be in the competition
of the application. The educational purpose and excitement of finding out the clues are not there.
But it's still a game collecting some sort of treasure/object.

%##############################################################################
\section{Stand out from the rest - Future work}
%##############################################################################
Today the application Treasure Hunt only counts in if the user is in the correct position and check if the
answer is correct. If each location also gives some information about it surrounding it would
also add some educational purpose. This would increase the market, to not only consist of
people, but also companies example Museums etc.
Even adding some sort augmented object would increase the experience for the user. However,
such a system would need even more study and testing. By adding an augmented object could
the user collect different object, which could be shown in there profile page. 
It should be noted that even if the application/game has a certain quality, the main experience
lies in the information given from the route/hunt.

\bibliographystyle{plain}
\bibliography{ref_bibtex}

\end{document}
